\documentclass[a4paper]{article}

\usepackage[swedish]{babel}
\usepackage[utf8]{inputenc}
\usepackage{graphicx}

\begin{document}
\begin{titlepage}
\centering
{\bfseries\huge Projektrapport DAT290}

\vspace{10mm}

{\Large Radiostyrd Bil, Grupp 07}

\vspace{20mm}

{\Large \itshape{Joakim Junttila, Johanna Gudmandsen, Gustav Holst,\\Henrik Klein Moberg, Anders Berggren Sjöblom, \\[1mm] Stanislaw Zwierzchowski, Carl Lundgren}}

\vspace{10mm}

%Vet ej vilket datum som ska stå
{DATUM}


\normalsize{
\begin{table}[b]
\centering
\begin{tabular}{|l|l|l|}  \hline
          & \bf Namn & \bf Datum   \\ \hline \hline
 Granskad & NAMN     & DATUM        \\ \hline
 Godkänd  & NAMN     & DATUM         \\ \hline
  \end{tabular}  
  \end{table}}
\end{titlepage}

\tableofcontents


\newpage
\section{Introduktion}

% Liten text om vad detär för något vi håller på med

% Typ RC bilar har funnits sisåhär länge. en lite historia "typ aneckdåt"

% flytande text under hela punkt ett. med bara styckes indelningar.

Radiostyrda bilar började produceras på 60-talet~\cite{RCHistory}.

\subsection{Syfte}

% Hur de styr vad som kan bli bättre. 

\subsection{Mål}

% Samma som projektplanen fast städa lite och göra lite snyggare

\subsection{Arbetsmetod}

% hur vi har genomfört projecktet

% hur vi mätte signalerna 
% parallel arbete (delat upp projektet i olika grupper löser olika uppgifter så att alla har någott att göra)

\section{Teknisk beskrivning}

\subsection{Teknisk bakgrund}


\subsection{Systemöversikt}


\subsection{Delsystem}




\section{Resultat}




\section{Slutsats}



\newpage
%To references 
\bibliographystyle{IEEEtran}
\bibliography{referenser}


\end{document}