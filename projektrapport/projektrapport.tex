\documentclass[a4paper]{article}

\usepackage[swedish]{babel}
\usepackage[utf8]{inputenc}
\usepackage{graphicx}

\begin{document}
\begin{titlepage}
\centering
{\bfseries\huge Projektrapport DAT290}

\vspace{10mm}

{\Large Radiostyrd Bil, Grupp 07}

\vspace{20mm}

{\Large \itshape{Joakim Junttila, Johanna Gudmandsen, Gustav Holst,\\Henrik Klein Moberg, Anders Berggren Sjöblom, \\[1mm] Stanislaw Zwierzchowski, Carl Lundgren}}

\vspace{10mm}

%Vet ej vilket datum som ska stå
{DATUM}


\normalsize{
\begin{table}[b]
\centering
\begin{tabular}{|l|l|l|}  \hline
          & \bf Namn & \bf Datum   \\ \hline \hline
 Granskad & NAMN     & DATUM        \\ \hline
 Godkänd  & NAMN     & DATUM         \\ \hline
  \end{tabular}  
  \end{table}}
\end{titlepage}

\tableofcontents


\newpage
\section{Introduktion}

Radiostyrda bilar med handkontroller började produceras på mitten av 60-talet~\cite{RCHistory}. Under åren har den båda tävlats med och varit en väletablerad leksak till barn och trots att den genomgått småjusteringar har den radiostyrda bilen i det stora hela förblivit densamma.

\subsection{Syfte}

Syftet med detta projekt är att uppdatera den klassiska radiostyrda bilen med hjälp av ny teknik och följaktligen få en mer modern teknisk produkt.

\subsection{Mål}

Projektet går ut på att ersätta en del av elektroniken i en radiostyrd bil. Det utgår från en bil där mottagare och sändare i den befintliga styrelektroniken byts ut mot ARM-baserade system. Vidare ska en kontrollapplikation implementeras med hjälp av ett antal avståndsmätare, därefter ska bilen självständigt kunna köra rakt fram i högsta möjliga takt och på ett avstånd av maximalt 1 cm från en vägg bromsa in helt utan att kollidera med väggen. Dessa uppgraderingar ska båda använda sig av Bluetooth. 

Sedan implementeras styrning via en telefonapplikation som ska kunna kontrollera bilen genom ett Androidsystem. Även denna uppgradering ska använda sig av Bluetooth.

\subsection{Arbetsmetod}

% hur vi har genomfört projecktet

% hur vi mätte signalerna 
% parallel arbete (delat upp projektet i olika grupper löser olika uppgifter så att alla har någott att göra)

\section{Teknisk beskrivning}

\subsection{Teknisk bakgrund}


\subsection{Systemöversikt}


\subsection{Delsystem}




\section{Resultat}




\section{Slutsats}



\newpage
%To references 
\bibliographystyle{IEEEtran}
\bibliography{referenserRapport}


\end{document}